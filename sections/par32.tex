\subsection{Хроматическое число графа}

\begin{defn}
    Хроматическое число $\chi(G)$: наименьшее число цветов, в которые можно правильно покрасить вершины графа $G$.
\end{defn}
Критерий раскарашиваемости в два цвета:

\begin{theorem}
    Граф двудолен если и только если он не содержит нечетных циклов.    
\end{theorem}

\begin{proof}
    
    "$\Rightarrow$":

    Начнем путь из левой доли. Нужно пройти по четному числу ребер чтобы вернуться в левую долю $\implies$ все циклы четные.

    "$\Leftarrow$":

    Пусть в графе $G$ нет нечетных циклов. Выберем произвольную вершину $u$ и разобьем множество вершин на два непересекающихся множества $U$ и $V$ так, чтобы в $U$ лежали вершины $v_0$, такие что кратчайшая цепь $(u, v_0)$ была четной длины, а в $V$ соответственно вершины $v_1$, для которых длина цепи $(u, v_1)$ --- нечетная. При этом $u \in U$. 

    В графе $G$ нет ребер $ab$, таких что $a, b$ лежат одновременно в $U$ и $V$. Докажем это от противного.

    Пусть $a, b \in U$. Зададим $P_0$ --- кратчайшая $(u, a)$ цепь, а $P_1$ --- кратчайшая $(u, b)$ цепь. Обе цепи четной длины. Пусть $v_0$ --- последняя вершина цепи $P_0$, принадлежащая $P_1$. 
    
    Тогда подцепи от $u$ до $v_0$ в $P_0$ и $P_1$ имеют одинаковую длину (иначе бы, пройдя по более короткой подцепи от $u$ до $v_0$ мы смогли бы найти более короткую цепь от $u$ до $a$ или от $u$ до $b$, чем цепь $P_0$ или $P_1$).

    Так как подцепи от $v_0$ до $a$ и от $v_0$ до $b$ в цепях $P_0$ и $P_1$ имеют одинаковую четность, а значит в сумме с ребром $ab$ они образуют цикл нечетной длины, что невозможно.

    Значит $G$ двудолен.
\end{proof}

\subsubsection*{Простые оценки}

\begin{lemma}
    \label{lem:geqk}
    Если граф $H$ нельзя покрасить в $k$ цветов, то он содержит индуцированный подграф, в котором степени вершин $\geq k$.
\end{lemma}

\begin{proof}
    Если $\deg{v} < k$, то граф $H \setminus v$ также нельзя покрасить в $k$ цветов(иначе покрасим его, а потом докрасим $v$). Удалим вершину $v$ и продолжим процесс, в итоге останется подграф, в котором все степени не меньше $k$.
\end{proof}

\begin{follow}
    Пусть $v_1, v_2, \ldots, v_n$ --- все вершины графа $G$, и при всех $k = 1, 2, \ldots, n$ вершина $v_k$ имеет не более чем $d$ соседей среди вершин $v_1, \ldots, v_{k-1}$. Тогда $\chi(G) \leq d + 1$.
\end{follow}

\begin{proof}
    Предположим противное, т.е. $\chi(G) > d + 1$.

    По лемме \ref*{lem:geqk} в $G$ найдется индуцированный подграф, степени вершин которого $\geq d + 1$.
    
    Но вершина этого подграфа с наибольшим номером имеет в нем степень не более $d$. Противоречие.
\end{proof}

\begin{follow}
    Если степени всех вершин графа $G$ не превосходят $d$, то $\chi(G) \leq d + 1$.
\end{follow}

\begin{proof}
    Пронумеруем вершины в произвольном порядке и применим предыдущее следствие.
\end{proof}