\subsection{Конъюктивная нормальная форма (КНФ)}

\begin{defn}
    Простая дизъюнкция --- дизъюнкция  одной или нескольких переменных или их отрицаний, причем все переменные встречаются не более одного раза.
\end{defn}

\begin{defn}
    Конъюктивная нормальная форма --- представление булевой функции в виде конъюнкции простых дизъюнкций. 
\end{defn}

\begin{example}
    $(x \vee \neg y) \wedge z$
\end{example}

\begin{defn}
    Совершенная КНФ --- КНФ, в любой конъюнкции которой участвуют все переменные.
\end{defn}

\subsubsection*{Построение СКНФ по таблице истинности}

\[f(x_1, \ldots, x_n) = \bigwedge_{f(\sigma_1, \ldots, \sigma_n) = 0}
(x_1^{\neg \sigma_1}\ \vee\  \ldots\ \vee\ x_n^{\neg \sigma_n})\]    