\subsection{Дизъюнктивная нормальная форма (ДНФ)}

\begin{theorem-non}
    \begin{equation*}
        x^{\sigma} = 
        \begin{cases}
            x, &\text{$\sigma = 1$}\\
            \neg x, &\text{$\sigma = 0$}
        \end{cases}
    \end{equation*}
\end{theorem-non}

\begin{defn}
    Простая конъюнкция --- конъюнкция одной или нескольких переменных или их отрицаний, причем все переменные встречаются не более одного раза.
\end{defn}

\begin{defn}
    Дизъюнктивная  нормальная форма --- преставление булевой функции в виде дизъюнкции простых конъюнкций. 
\end{defn}

\begin{example}
    $(x \wedge \neg y) \vee z$
\end{example}

\begin{defn}
    Совершенная ДНФ --- ДНФ, в любой конъюнкции которой участвуют все переменные.
\end{defn}

\subsubsection*{Построение СДНФ по таблице истинности}

\[f(x_1, \ldots, x_n) = \bigvee_{f(\sigma_1, \ldots, \sigma_n) = 1} 
(x_1^{\sigma_1}\ \wedge\  \ldots\ \wedge\ x_n^{\sigma_n})\]

\begin{theorem}
    Для любой булевой функции, не равной тождественно нулю, существует СДНФ ее задающая.
\end{theorem}

\begin{proof} 
    $(x_1^{\sigma_1}\ \wedge\ , \ldots,\ \wedge\ x_n^{\sigma_n}) = 1 \iff \forall_{i=1}^{n} \ x_i^{\sigma_i} = 1 \iff \forall_{i=1}^{n} \ x_i = \sigma_i$ по обозначению $x^{\sigma}$. 
    Булева функция не равна 0 $\iff \exists\ (x_1^{\sigma_1} \ \wedge\ , \ldots,\ \wedge\ x_n^{\sigma_n}) = 1 \iff$ для некоторого набора $(\sigma_1, \ldots, \sigma_n)\ \forall_{i=1}^{n}\ x_i = \sigma_i \iff f(x_1, \ldots, x_n) = 1$ по построению. 
\end{proof}