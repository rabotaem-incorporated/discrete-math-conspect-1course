\begin{normalsize}

\subsection{Паросочетания в графах общего вида}

\begin{notice}
    КС --- компонента связности.

    нечетная КС --- КС с нечетным числом вершин.

    Для $U \subseteq V$ обозначим $G \setminus U$ индуцированный подграф на $V \setminus U$.
\end{notice}

\begin{theorem}[Теорема 2, Татт, 1947]
    В графе $G = (V, E)$ есть совершенное паросочетание $\iff \forall U \subseteq V$ подграф $G \setminus U$ содержит не более $|U|$ нечетных КС.

    В частности, условие для $U = \emptyset$ означает, что $|V|$ четно.
\end{theorem}

\begin{proof}
    
    "$\Rightarrow$":

    Пусть $M \subseteq E$ --- совершенное паросочетание, и пусть $U \subseteq V$ --- подмножество вершин.

    Тогда в $G \setminus U$ для всякой нечетной КС $C \subseteq V \setminus U$ паросочетание $M$ должно содержать хотя бы одно ребро между $C$ и $U$, т.е., $(u_C, v_C)$, где $u_C \in U$ и $v_C \in C$.

    Так как вершины $u_C$, выбранные для разных таких компонент $C$, повторяться не могут (тогда это не было бы паросочетанием), получается, что число вершин в $U$ не может быть меньше, чем число нечетных компонент связности.

    "$\Leftarrow$":

    Пусть совершенного паросочетания нет.

    Пусть $\hat{G} = (V, \hat{E})$ граф, полученный из $G$ добавлением максимального числа ребер, так, чтобы в нем все еще не было совершенного паросочетания, но добавление любого дополнительного ребра приводило бы к появлению такового.

    Тогда достаточно построить $U \subseteq V$, удаление которого разбивало бы $\hat{G}$ так, чтобы в нем оставалось более чем $|U|$ нечетных КС --- тогда и в $G$ число нечетных КС будет не меньше (удаление одного ребра либо сохраняет нечетную КС, либо разбивает ее на две, одна из которых опять нечетная).

    $U := \{v \in V \mid \deg v = |V| - 1\}$.
    
    \begin{theorem-non}
        В $\hat{G} \setminus U$ всякая КС --- полный граф.
    \end{theorem-non}
    
    \begin{proof}
        
        Пусть есть КС $C \subseteq V \setminus U$, которая не является полным графом.
        
        Т.е. сущестуют вершины $v_1, v_2, v_3 \in C$, для которых $(v_1, v_2), (v_1, v_3) \in \hat{E}, (v_2, v_3) \notin \hat{E}$.
        
        Т.к. $v_1 \notin U$, то $\exists v_4 \in V: (v_1, v_4) \notin \hat{E}$.
        
        $\hat{G} \implies$ если добавить в него ребро $(v_1, v_4)$, то будет совершенное паросочетание $M_1 \subseteq \hat{E} \cup \{(v_1, v_4)\}$.
        
        Но раз в $\hat{G}$ совершенного паросочетания не было, то $(v_1, v_4) \in M_1$.
        
        Аналогично при добавлении ребра $(v_2, v_3)$ получится совершенное паросочетание $M_2 \subseteq \hat{E} \cup \{(v_2, v_3)\}$, где $(v_2, v_3) \in M_2$.
        
        $G' := (V, M_1 \cup M_2)$ состоит из отдельных ребер из $M_1 \cap M_2$, а также из циклов четной длины, в которых чередуются ребра из $M_1$ и $M_2$.
        
        Ребра $(v_1, v_4)$ и $(v_2, v_3)$ попадут в такие циклы, поскольку каждое из них принадлежит ровно одному из двух паросочетаний.
        
        Рассмотрим два случая.
        
        \begin{itemize}
            \item Если эти ребра попадают в один и тот же цикл, то его можно перестроить, задействовав одно из ребер $(v_1, v_2)$ и $(v_1, v_3)$ --- получим совершенное паросочетание для $\hat{G}$.
        
            \item Если же эти ребра попадают в разные циклы, то каждом цикле можно взять другие ребра, и опять получится совршенное паросочетание для $\hat{G}$.
        \end{itemize}
        
        Утверждение доказано(продолжим доказывать теорему).
    \end{proof}
    
    Итак, удалением $U \subseteq V$ получаются КС --- полные графы, из них не более $|U|$ нечетных.
    
    Строим совершенное паросочетание в $\hat{G}$: четные КС сами с собой; нечетные --- соединением одной вершины с произвольной вершиной из $U$, остальные вершины --- сами с собой; оставишиеся вершины из $U$ --- между собой.
    
    Противоречие.
\end{proof}

\end{normalsize}