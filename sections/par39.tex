\subsection{Связность и разделяющие множества}

\begin{defn}
    Пусть $V_1, V_2 \subseteq V(G)$. Множество $X \subseteq V(G)$ называется $(V_1, V_2)$ - разделяющим, если в графе $G \setminus X$ нет путей из $V_1$ в $V_2$.
\end{defn}

\begin{theorem}[Геринг, 2000]
    Пусть $V_1, V_2 \subseteq V(G),~k \in \N$ натуральное число. Тогда верно ровно одно из двух условий:

    \begin{enumerate}
        \item В $V(G)$ найдется подмножество $U,~|U| < k$, разделяющее $V_1$ и $V_2$;
        \item В $G$ найдется не менее $k$ простых путей из $V_1$ в $V_2$, попарно не имеющих общих вершин.
    \end{enumerate}
\end{theorem}

\begin{proof}

    Понятно, что 1 и 2 одновременно выполняться не могут:

    Разделяющее множество обязано содержать хотя бы по одной вершине из каждого из путей из $V_1$ в $V_2$.

    Таким образом, требуется доказать не 1 $\implies$ 2 --- то есть, если любое $(V_1, V_2)$ --- разделяющее множество содержит $\geq k$ вершин, то найдутся $k$ путей из $V_1$ в $V_2$.

    Индукция по $|V|$.

    \textsl{База:} $|V| = 1$ очевидна.

    \textsl{Переход:}

    Будем удалять ребра до тех пор, пока любое $(V_1, V_2)$ --- разделяющее множество содержит $\geq k$ вершин. Когда-то это закончится (если только $|V_1 \cap V_2| < k$ --- но если $|V_1 \cap V_2| \geq k$, то имеется $k$ одновершинных путей из $V_1$ в $V_2$).

    Итак, при удалении ребра xy образуется $(V_1, V_2)$ --- разделяющее множество $Z,~|Z| < k$.

    Заметим, что множество $Z \cup x$ было разделяющим и до удаления ребра xy, а тогда $|Z| = k - 1,~|Z \cup x| = k$.

    Аналогично для $Z \cup y$.

    Два случая:

    \textbf{Случай 1:}

    Одно из множеств $Z \cup x$ и $Z \cup y$ совпадает с $V_1$, а второе с $V_2$. В качестве $k$ путей из $V_1$ в $V_2$ можно взять вершины $Z$ и ребро xy.

    \textbf{Случай 2:}

    Одно из множеств $Z \cup x$ и $Z \cup y$ отлично и от $V_1$, и от $V_2$. Обозначим это множество $W$, тогда $|W| = k$, $W \neq V_1,~W \neq V_2$ и $W$ --- $(V_1, V_2)$ --- разделяющее множество в нашем графе.

    Заметим, что никакой путь из $V_1$ в $W$ не проходит через вершины (непустого!) множества $V_2 \setminus W$ --- иначе бы $W$ не разделяло $V_1$ и $V_2$.
    
    Выкинем из нашего графа множество вершин $V_2 \setminus W$ --- обозначим новый граф $G_1$.

    Заметим, что любое $(V_1, W)$ --- разделяющее множество в $G_1$ является $(V_1, W)$ --- разделяющим и в старом, поскольку то, что мы выкинули, никак не помогает добраться из $V_1$ в $W$.
    
    Следовательно, оно является и $(V_1, V_2)$ --- разделяющим, ибо любой путь из $V_1$ в $V_2$ заходит в $W$.

    Поэтому в нем не менее $k$ вершин.

    Но $|V(G_1)| < |V(G)| \implies$ по предположению индукции имеется $k$ непересекающихся путей из $V_1$ в $W$.

    Аналогично, имеется $k$ непересекающихся путей из $W$ в $V_2$.

    Заметим, что пути из $V_1$ в $W$ и из $W$ в $V_2$ не могут пересекаться, кроме как по общему концу в $W$ --- это бы означало, что $W$ не разделяет $V_1$ и $V_2$.
    
    Склеим два наших набора по $k$ путей $\implies$ получим $k$ непересекающихся путей из $V_1$ в $V_2$.
\end{proof}

\begin{theorem}[Менгер, 1927]
    Пусть вершины $a$ и $b$ связного графа $G$ не соединены ребром. Тогда наименьшее число вершин $(a, b)$-разделяющего множества равно наибольшему числу непересекающихся по вершинам путей, соединяющих $a$ и $b$.

    В формулировке теоремы подразумевается, что разделяющее множество не содержит $a$ и $b$, а пути не пересекаются по вершинам, не являющимся начальной или конечной.
\end{theorem}

\begin{proof}
    Достаточно рассмотреть граф $G - a - b$ и применить теорему Геринга к множествам $V_1, V_2$, где $V_1$ --- множество соседей $a$, $V_2$ --- множество соседей $b$. (а $k$ --- наименьшая мощность $(V_1, V_2)$-разделяющего множества).
\end{proof}