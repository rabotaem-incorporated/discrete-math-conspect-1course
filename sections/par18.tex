\subsection{Основные определения}

\begin{defn}
    Графом называется пара $G = (V, E)$, где $V$ --- конечное множество вершин, а $E \subseteq V \times V$ --- множество ребер.
\end{defn}

\begin{defn}
    Граф можно задать матрицей смежности $A = (a_{ij})$ порядка $|V|$:
    \[ 
        a_{ij} = \begin{cases}
            1, & \text{если $(i, j) \in E$,}\\
            0, & \text{иначе.}
        \end{cases}
    \]
\end{defn}

\begin{defn}
    Граф неориентированный, если $(u, v) \in E \iff (v, u) \in E$. Иначе граф называется ориентированным (орграф). Если не указано, что граф ориентированный, то подразумевается что он неориентированный.
\end{defn}

\begin{defn}
    Мультиграф: допускаются кратные ребра (в матрице смежности соответствует натуральным числам).
\end{defn}

\begin{defn}
    Две вершины $v, u$ называются смежными, если $(u, v) \in E$.
\end{defn}

\begin{defn}
    Вершина $v$ и ребро $e$ называются инцидентными, если $e = (v, u)$ для некоторой вершины $u$.
\end{defn}

\begin{defn}
    Ребро, концевые вершины которого совпадают, называется петлей.
\end{defn}

\begin{defn}
    Степень $\deg(v)$ вершины $v$ --- число инцидентных ей ребер (петля считается дважды).
\end{defn}

\begin{lemma} ~

    \begin{enumerate}
        \item Во всяком графе сумма степеней всех вершин равна удвоенному числу ребер: $\sum\limits_{v \in V} \deg(v) = 2|E|$.
        
        \item В ориентированном графе сумма входящих степеней равна сумме исходящих степеней.
        
        \item Всякий конечный граф содержит четное число вершин нечетной степени.
    \end{enumerate}

\end{lemma}

\begin{proof}
    \begin{enumerate}
        \item Каждое ребро инцидентно двум вершинам, поэтому его удаление не уменьшает сумму степеней всех вершина на $2$. Удаляя по очереди все ребра (пусть их $k$), придем к пустому графу, в котором сумма степеней равна $0$. Значит вначале она была равна $2k$.
        
        \item В ориентированном случае при удалении ребра как сумма входящих, так и сумма исходящих степеней уменьшается на $1$, откуда аналогично следует второе утверждение леммы. 
        
        \item Получено, что сумма степеней вершин четна. А для этого необходимо, чтобы нечетных слагаемых было четное число.
    \end{enumerate}
\end{proof}

\subsubsection*{Пути и циклы}

\begin{defn}
    Путь, соединяющий вершины $v_0$ и $v_n$: последовательность вершин и ребер $v_0 \overset{e_1}{\to} v_1 \overset{e_2}{\to} \dots \overset{e_n}{\to} v_n$ из $v_0$ в $v_n$, так что $e_i = (v_{i - 1}, v_i) \in E$ для всех $i = 1, 2, \ldots, n$.
\end{defn}

\begin{defn}
    Если все вершины пути различны, то он называется простым;
\end{defn}

\begin{defn}
    Если все ребра пути различны, то он называется реберно-простым;
\end{defn}

\begin{defn}
    Реберно-простой путь с $v_0 = v_n$ называется циклом.
\end{defn}

\begin{defn}
    Цикл называется простым, если различны вершины $v_0, v_1, \ldots, v_{n - 1}$.
\end{defn}

\begin{notice}
    Если между двумя вершинами есть путь, то есть и простой путь. В частности, если в графе есть цикл, то есть и простой цикл.
\end{notice}

\subsubsection*{Связность}

\begin{defn}
    Если две вершины неориентированного графа совпадают или соединены некоторым путем, они называются связанными.

    В ориентированном случае связанными называются такие вершины $a$ и $b$, что существует пути как из $a$ в $b$, так и из $b$ в $a$ (либо $a = b$).
\end{defn}


\subsubsection*{Отношение эквивалентности}

\begin{defn}
    Бинарное отношение на множестве $X$ --- это подмножество $X \times X$.
\end{defn}

\begin{defn}
    Отношение эквивалентности $\sim$ на множестве $X$ --- это бинарное отношение, для которого выполнены следующие условия:

    \begin{itemize}
        \item Рефлексивность: $a \sim a$ для любого $a \in X$
        \item Симметричность: $a \sim b \iff b \sim a$
        \item Транзитивность: $a \sim b$ и $b \sim c$ $\implies$ $a \sim c$
    \end{itemize}
\end{defn}

\subsubsection*{Классы эквивалентности}

Для каждого $x \in X$ определим класс $C_x = \{y \in X \mid y \sim x\}$.

\begin{theorem-non}
    $X$ разбивается на (непересекающиеся) классы эквивалентности.
\end{theorem-non}

\begin{proof}

    Рефлексивность $\implies x \in C_x$;

    Cимметричность $\implies x \in C_y \iff y \in C_x$;

    Транзитивность $\implies y \in C_x \implies C_y \subseteq C_x$. (действительно, для всякого $z \in C_y$ имеем $z \sim y \sim x \implies x \sim z \iff z \in C_x$ 

    Меняя $x$ и $y$ местами, получаем $C_x \subseteq C_y$, то есть $C_x = C_y$.

    Наконец, если $C_x$ и $C_y$ пересекаются, $z \in C_x \cap C_y$, то по доказанному выше $C_x = C_z = C_y$.
\end{proof}

\subsubsection*{Компоненты связности}

\begin{defn}
    Связанность --- отношение эквивалентности на множестве вершин.
\end{defn}

\begin{defn}
    Классы эквивалентности называются компонентами связности (в ориентированном случае иногда говорят компоненты сильной связности).
\end{defn}

\begin{defn}
    Граф связный, если в нем ровно одна компонента связности.
\end{defn}

\begin{defn}
    Орграф, в котором одна компонента связности называют сильно связным.
\end{defn}

\begin{statement}
    Компонента связности является связным графом. Компонента связности орграфа является сильно связным орграфом.
\end{statement}

\begin{proof}

    Для вершин $u, v$ одной компоненты связности (или сильной связности для орграфа) есть путь $P$ из $u$ в $v$ в исходном графе, а доказать надо, что есть путь в компоненте. Это следует из того, что любая промежуточная вершина пути $P$ в исходном графе связана как с $u$, так и с $v$, так что все они действительно лежат в компоненте связности.

\end{proof}

В неориентированном случае между вершинами из разных компонент связности ребер нет. В ориентированном случае все ребра между вершинами двух компонент $A$ и $B$ направлены в одну сторону (либо все из $A$ в $B$, либо все из $B$ в $A$).