\subsection{Теорема Петерсена}

\begin{defn}
    $k$-регулярный граф --- степень каждой вершины равна $k$.
\end{defn}

\begin{theorem}[Петерсен, 1891]
    Во всяком $3$-регулярном графе без мостов есть совершенное паросочетание.
\end{theorem}

\begin{proof}
    Для всякого множества вершин $U \subseteq V$ рассмотрим подграф $G \setminus U$, и в нем все нечетные КС $C_1, \ldots, C_k$.

    Докажем утверждение: каждая из этих КС соединена с $U$ в исходном графе $G$ нечетным числом ребер, и не менее чем тремя.

    Так как в КС нечетное число вершин, и все они нечетной степени, сумма их степеней нечетна. Из них четное число приходится на внутренние ребра, а оставшееся нечетное число --- на внешние. Поскольку каждое ребро входит в цикл (так как нет мостов), ребро не может быть единственным. Таких ребер нечетное количество, иначе сумма степеней снова будет нечетна. Значит их не менее трех. Утверждение доказано.
    
    Сумма степеней вершин из $|U|$ равна $3|U|$, и потому ребер, соединяющих $|U|$ с нечетными компонентами связности, всего не более чем $3|U|$.
    
    Так как в каждую нечетную КС идет не менее трех ребер, всего этих компонент не более чем $|U|$. По теореме Татта есть совершенное паросочетание.
\end{proof}