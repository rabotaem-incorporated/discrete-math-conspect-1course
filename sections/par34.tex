\subsection{Паросочетания в двудольном графе}

\begin{defn}
    Паросочетание(matching) --- подмножество ребер $M \subseteq E$, где никакие два ребра не имеют общих концов.
\end{defn}

\begin{defn}
    Совершенное паросочетание: учавствуют все вершины.
\end{defn}

\begin{theorem}[Теорема Холла, 1935]

    Пусть $G = (V_1, V_2, E)$ --- двудольный граф. 

    Паросочетание, покрывающее $V_1$, существует $\iff \forall U \subseteq V_1,~|U| = k$, у вершин $U$ в совокупности есть не менее $k$ смежных вершин в $V_2$.
\end{theorem}

\begin{proof}

    "$\Rightarrow$": 
    
    Очевидно: Паросочетание есть, значит $\forall U \subseteq V_1$ есть не менее $|U|$ смежных вершин в $V_2$.
    
    "$\Leftarrow$": 
    
    Индукция по $|V_1|$.
    
    База: $|V_1| = 1$, и у единственного подмножества размера $1$ есть одна смежная вершина в $V_2$ --- это ребро и дает паросочетание.
    
    Переход: 
    
    Случай 1: Пусть есть подмножество $U_1 \subset V_1,~|U_1| = k$, у которого ровно $k$ смежных вершин, и пусть $U_1 \subseteq V_2$ --- смежные с ними вершины, где $|U_2| = |U_1|$.
    
    В подграфе на вершинах из $U_1$ и $U_2$ у каждого подмножества $U_1$ размера $m$ есть не менее $m$ соседок из $V_2$, и, следовательно, из $U_2$. Тогда, по предположению индукции, есть паросочетание, покрывающее $U_1$.
    
    Покажем, что в подграфе на вершинах из $V_1 \setminus U_1$ и $V_2 \setminus U_2$ также выполняется условие теоремы, то есть у всякого подмножества $V_1 \setminus U_1$ размера $l$ есть не менее чем $l$ смежных вершин в $V_2 \setminus U_2$.
    
    Пусть $W \subseteq V_1 \setminus U_1$ --- любое подмножество. Тогда подмножество $U_1 \cup W$ имеет не менее чем $k+l$ смежных вершин по условию.
    
    При этом у вершин из $U_1$ всего $k$ смежных вершин, которые также смежны с $U_2$, и, --- следовательно, остальные $l$ смежных вершин смежны с $W$ и лежат вне $U_2$.
    
    Следовательно, условие выполняется, и, по предположению индукции, есть паросочетание, покрывающее $V_1 \setminus U_1$, которое не пересекается с ранее построенным паросочетанием, покрывающим $U_1$.
    
    Случай 2: У всякого подмножества $U_1 \subset V_1$ есть не менее чем $|U_1| + 1$ смежных вершин.
    
    Пусть $(v_1, v_2) \in E$ --- произвольное ребро. Подграф, образованный удалением вершин $v_1$ и $v_2$, продолжает удовлетворять условию теоремы, поскольку в нем у каждого подмножества $U_1 \subseteq V_1 \setminus \{v_1\}$ остается не менее чем $(|U_1| + 1) - 1$ смежных вершин, за возможной потерей $v_2$.
    
    Следовательно, по предположению индукции, в нем есть паросочетание, покрывающее $V_1 \setminus \{v_1\}$. Возвращая $v_1, v_2$ и ребро $(v_1, v_2)$, получаем искомое паросочетание.
\end{proof}