\subsection{Остовное дерево}

\begin{defn}
    $H$ --- остовный подграф $G$, если $V(H) = V(G)$.
\end{defn}

\begin{defn}
    Остовное дерево --- остовный подграф, который является деревом.
\end{defn}

\begin{statement}
    Всякий связный граф содержит остовное дерево.
\end{statement}

\begin{proof}
    
    Если связный граф $G$ не содержит циклов, то он сам является своим остовным деревом.

    В противном случае выберем произвольное ребро $e$ графа $G$, входящее в цикл, и удалим его из $G$ --- связность сохраняется. Будем повторять процедуру удаления ребра из цикла, пока не получим связный граф без циклов.

\end{proof}

\begin{follow}
    В связном графе с $n$ вершинами хотя бы $n - 1$ ребро.
\end{follow}