\subsection{Несовершенные паросочетания}

\begin{notice}
    $odd(G)$ --- число нечетных КС в $G$.
\end{notice}

\begin{theorem}[Формула Бержа, 1958]
    Число вершин, непокрытых наибольшим паросочетанием, равно
    \[
    \max\limits_{U \subseteq V} (odd(G \setminus U) - |U|).   
    \]
    Эта величина иногда называется дефектом $d(G)$ графа $G$.
\end{theorem}

\begin{notice}
    $d(G) = 0$ соответствует теореме Татта.
\end{notice}

\begin{proof}

    "$\geq$":

    Аналогично доказательству простой части теоремы Татта:
    Пусть $M \subseteq E$ --- паросочетание, и пусть $U \subseteq V$ --- подмножество ввершин, для которого достигается максимум $(odd(G \setminus U) - |U|)$.

    В $G \setminus U$ во всякой нечетной КС $C \subseteq V \setminus U$ есть
    \begin{itemize}
        \item или вершина, не покрытая паросочетанием $M$,
    
        \item или вершина $v_c \in C$, для которой паросочетание $M$ содержит ребро $(u_C, v_C)$, где $u_C \in U$.
    \end{itemize}

    Вершины $u_C$ для разных таких КС $C$ не повторяются.

    Отсюда нечетных КС, в которых есть непокрытая вершина не менее чем $(odd(G \setminus U) - |U|)$.

    "$\leq$":

    Пусть $k = \max\limits_{U \subseteq V} (odd(G \setminus U) - |U|)$.

    В граф добавляем $k$ новых вершин $\{v_1, \ldots, v_k\}$ и соединяем ребрами со всеми вершинами из $V$. Покажем, что полученный граф $G'$ удовлетворяет условию теоремы Татта.

    Для всякого $U' \subseteq V \cup \{v_1, \ldots, v_k\}$ рассмотрим два случая:

    \begin{itemize}
        \item если не все вершины $\{v_1, \ldots, v_k\}$ попали в $U'$, то после удаления $U'$ останется связный граф(т.е., не более $1$ нечетной КС).
        
        При $|U'| = \emptyset$ получается четная КС, так как $k$ и $|V|$ одной четности.
        
        \item если в $\{v_1, \ldots, v_k\} \subseteq U'$ попали все новые вершины, то по сути из исходного графа $G$ удаляются $|U'| - k$ вершин.
        
        Оценим число образующихся нечетных КС:
        
        $odd(G' \setminus U') - (|U'| - k) \leq k$;
        
        $odd(G' \setminus U') \leq |U'|$.
        
        Тогда, по теореме Татта, существует совершенное паросочетание в $G'$. После удаления из графа дополнительных вершин останется не более чем $k$ вершин, не покрытых этим паросочетаним.
    \end{itemize}
\end{proof}