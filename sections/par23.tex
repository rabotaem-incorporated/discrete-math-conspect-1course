\subsection{Теорема о деревьях}

\begin{defn}
    Простой граф --- неориентированный граф без петель и кратных ребер.
\end{defn}

\begin{theorem}
    Для простого графа $G$ следующие условия эквивалентны:

    \begin{enumerate}
        \item $G$ --- дерево;
        
        \item любые две различные вершины в $G$ соединены ровно одним простым путем;
        
        \item $G$ не содержит циклов, но если любую пару несмежных в $G$ вершин соединить ребром, то в полученном графе будет ровно один цикл;

        \item $G$ --- связный граф и $|V| = |E| + 1$;

        \item $G$ не содержит циклов и $|V| = |E| + 1$;

        \item $G$ --- связный граф, и всякое ребро в $G$ является мостом.
    \end{enumerate}
\end{theorem}


\begin{proof}
    Предположим $G$ --- дерево и докажем условия $(2)$-$(6)$.

    \textbf{2.} Дерево --- связный граф, а значит, любые две вершины соединены путем.

    Предположим, что вершины $u$ и $v$ соединены в $G$ не менее чем двумя цепями. Пусть 
    \begin{gather*}
        u = v_0 \to v_1 \to \cdots \to v_k = v \text{ и}\\
        u = v'_0 \to v'_1 \to \cdots \to v'_{l} = v.
    \end{gather*}
    
    различные простые пути из $u$ в $v$.

    Поскольку первые вершины в этих цепях совпадают, существует число $i$ такое, что $v_0 = v'_0, \ldots, v_i = v'_i$, но $v_{i + 1} \neq v'_{i + 1}$.

    Пусть $j$ --- наименьшее из чисел, больших $i$, такое, что вершина $v_j$ принадлежит второй цепи (такое $j$ существует, поскольку в рассматриваемых цепях совпадают и последние вершины).

    Тогда путь $v_i \to \ldots v_j = v'_r \to \ldots v'_i = v_i$ не содержит повторяющихся ребер, а значит, является циклом в $G$, противоречие.

    \textbf{3.} При добавлении к простому пути из $u$ в $v$ ребра $(u, v)$ очевидно, возникает цикл. Таким образом, из связности $G$ следует, что цикл возникает при добавлении любого ребра. Если при добавлении ребра $(u, v)$ возникло более одного цикла, значит вершины $u$ и $v$ соединены более чем одной цепью, что невозможно, так как это противоречит условию $(2)$.

    \textbf{4, 5.} Индукция по числу вершин в графе.

    \textsl{База:} $|V| = 1$, тогда $|E| = 0$, равенство верно.

    \textsl{Переход:} Пусть $|V| > 2$. Сперва покажем, что в графе есть вершина степени $1$. Вершин степени $0$ нет, потому что граф связный. Если каждая вершина имеет степень $2$ и более, то можно построить цикл, двигаясь из вершины в вершину (используя конечность графа). Следовательно, есть вершина $v$ степени $1$. Если удалить эту вершину и инциндентное ей ребро, получится дерево с $|V| - 1$ вершинами и $|E| - 1$ ребрами. По предположению индукции для него верно $|V| - 1 = |E| - 1 + 1$, и отсюда $|V| = |E| + 1$.
    
    \textbf{6.} По теореме о мостах.

    Теперь покажем, что каждое из условий $(2)$-$(6)$ влечет, что $G$ --- дерево.

    \textbf{2.} Поскольку любые две вершины соединены простым путем, $G$ --- связен, а так как цепь единственна, то в $G$ нет циклов (две вершины, находящиеся в цикле, соединены по крайней мере двумя цепями --- фрагментами этого цикла).

    \textbf{3.} Циклов в $G$ нет по условию. Предположим, что в $G$ более одной компоненты связности. Соединим ребром две вершины из разных компонент. В полученном графе новое ребро будет мостом по определению. По теореме о мостах оно не лежит ни в каком цикле, то есть при его добавлении цикл не образовался, противоречие.

    \textbf{4.} Вначале докажеи, что в связном графе $|V| \leq |E| - 1$. 

    Возьмем граф без ребер с $n$ вершинами и будем добавлять ребра по одному.

    Если добавленное ребро в новом графе оказалось мостом, то новый граф содержит ровно на одну компоненту связности меньше, чем старый.

    Если же добавленное ребро --- не мост, то число компонент связности не изменилось.

    Поскольку в исходном графе $n$ компонент связности, необходимо как минимум $n - 1$ ребер, чтобы сделать его связным.

    Граф $G$ связен по условию. Если в нем есть цикл, удалим из него ребро и получим связный граф, у которого ребер на $2$ меньше, чем вершин, что невозможно, как мы только что доказали.
    
    \textbf{5.} Так как $G$ не содержит циклов, каждая из его компонент связности является деревом, а значит, по доказанному ранее, число ребер в ней на единицу меньше числа вершин. Поскольку это же условие выполняется и для всего графа, компонента связности может быть только одна.

    \textbf{6.} Связность $G$ дана по условию, а отсутствие циклов прямо следует из теоремы о мостах.

\end{proof}